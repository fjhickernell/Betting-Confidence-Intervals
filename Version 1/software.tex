\documentclass{article}
\usepackage{amsmath,amsfonts,amssymb,amsthm}
\usepackage{natbib}
\usepackage{booktabs}
\usepackage{xcolor}
\usepackage{paralist}
\usepackage{xspace}
\usepackage{graphicx}
\usepackage[hyphens]{url}
\usepackage{hyperref}
%\usepackage{ulem}  % this underlines titles in the references so we should comment it out before the final version
\def\UrlBreaks{\do\/\do-}
\usepackage[paperwidth=8.5in, textwidth=6in]{geometry}
\usepackage{setspace}

\newtheorem{theorem}{Theorem}

\newcommand{\art}[1]{\begingroup\color{blue}#1\endgroup}
\newcommand{\aadit}[1]{\begingroup\color{orange}#1\endgroup}
\newcommand{\fred}[1]{\begingroup\color{red}#1\endgroup}
\newcommand{\aleksei}[1]{\begingroup\color{green}#1\endgroup}
\usepackage[normalem]{ulem} % adds the \sout command

% Lets make it look like writing on a blackboards
% not a 1970s typewriter!
\renewcommand{\le}{\leqslant}
\renewcommand{\ge}{\geqslant}
\renewcommand{\leq}{\leqslant}
\renewcommand{\geq}{\geqslant}
\renewcommand{\emptyset}{\varnothing}

\input{FJHDef}

\newcommand{\real}{\mathbb{R}}
\newcommand{\tran}{\mathsf{\scriptsize T}}

\newcommand{\e}{\mathbb{E}}
\newcommand{\bsa}{\boldsymbol{a}}
\newcommand{\bsx}{\boldsymbol{x}}
\newcommand{\bsz}{\boldsymbol{z}}
\newcommand{\bsone}{\boldsymbol{1}}
\newcommand{\bszero}{\boldsymbol{0}}

\newcommand{\simiid}{\stackrel{\mathrm{iid}}{\sim}}
\newcommand{\toas}{\stackrel{\mathrm{a.s.}}{\to}}

\newcommand{\dunif}{\mathbb{U}}
\newcommand{\dnorm}{\mathcal{N}}

\newcommand{\giv}{\!\mid\!} % 

\newcommand{\prpl}{\text{PrPl}}
\newcommand{\prplh}{\text{PrPl-H}}
\newcommand{\prpleb}{\text{PrPl-EB}}
\newcommand{\prplpm}{\text{PrPl}\pm}
\newcommand{\eb}{\mathrm{E}}
\newcommand{\ben}{\mathrm{Ben}}

\newcommand{\ebi}{\mathrm{EBI}}
\newcommand{\hbi}{\mathrm{HBI}}

% commands in lower case give less carpal trouble
\newcommand{\mc}{\mathrm{MC}}
\newcommand{\rqmc}{\mathrm{RQMC}}
\newcommand{\hk}{\mathrm{HK}}

\newcommand{\rd}{\,\mathrm{d}}
\newcommand{\phz}{\phantom{0}}

\newcommand{\jmp}{\mathrm{jump}}
\newcommand{\knk}{\mathrm{kink}}
\newcommand{\smo}{\mathrm{smooth}}
\newcommand{\fin}{\mathrm{finance}}

\title{Reproducing the simulations in ``Empirical Bernstein and betting confidence intervals for randomized quasi-Monte Carlo''}
\date{April 2025}
\author{Aadit Jain\\Rancho Bernardo High School
 \and
 Fred J. Hickernell \\Illinois Institute of Technology
 \and
 Art B. Owen \\ Stanford University
 \and
 Aleksei G. Sorokin \\ Illinois Institute of Technology
}
\begin{document}
\doublespacing

\maketitle
Here is information regarding the simulations for the paper ``Empirical Bernstein and betting confidence intervals for randomized quasi-Monte Carlo'':
\begin{itemize}
    \item The simulations, the data generated from the simulations, and the instructions to run the simulations are at the GitHub Repository \url{https://github.com/aaditj1962161/Betting-Paper-Simulations-for-QMC}.
    \item The simulations were run in the Jupyter notebook
\end{itemize}
were run in the Jupyter notebook at \url{https://github.com/aaditj1962161/Betting-Paper-Simulations-for-QMC/blob/paper-simulations/simulations/Fixed-time/Betting\%20IID\%20vs\%20QMC.ipynb}. The data from these simulations is at the csv file \url{https://github.com/aaditj1962161/Betting-Paper-Simulations-for-QMC/blob/paper-simulations/simulations/Fixed-time/qmc_combined_results.csv}. The instructions to run the simulation notebook are at the README file \url{https://github.com/aaditj1962161/Betting-Paper-Simulations-for-QMC/blob/paper-simulations/simulations/Fixed-time/README.md}.
\bibliographystyle{plain}
\bibliography{FJH25copy,FJHown25copy,ebci4rqmc}

\end{document}
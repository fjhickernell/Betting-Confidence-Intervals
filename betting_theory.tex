\documentclass{article}
\usepackage{amsmath,amsfonts,amssymb}
\usepackage{natbib}
\usepackage{booktabs}
\usepackage{xcolor}

\newcommand{\art}[1]{\begingroup\color{blue}#1\endgroup}
\newcommand{\aadit}[1]{\begingroup\color{orange}#1\endgroup}
\newcommand{\fred}[1]{\begingroup\color{red}#1\endgroup}
\newcommand{\alexei}[1]{\begingroup\color{green}#1\endgroup}

% Lets make it look like writing on a blackboards
% not a 1970s typewriter!
\renewcommand{\le}{\leqslant}
\renewcommand{\ge}{\geqslant}
\renewcommand{\leq}{\leqslant}
\renewcommand{\geq}{\geqslant}
\renewcommand{\emptyset}{\varnothing}
\newcommand{\e}{\mathbb{E}}
\newcommand{\var}{\mathrm{Var}}

% commands in lower case give less carpal trouble
\newcommand{\mc}{\mathrm{MC}}
\newcommand{\rqmc}{\mathrm{RQMC}}
\newcommand{\runif}{\mathbb{U}}
\newcommand{\rd}{\,\mathrm{d}}
\newtheorem{theorem}{Theorem}
\renewcommand{\thetheorem}{3} 
\begin{document}
\section{Introduction}
This document summarizes some results from \cite{WauRam24a}, which we can possibly analyze in the way the Bennett inequality was analyzed. The \textbf{[PrPl-EB-CI]} and \textbf{[Hedged-CI]} described in the following sections has been used in the simulations I've done till now.
\section{The EB CS and CI}
The EB CI described in the paper is tighter than that of Maurer and Pontil. The CI widths converge exactly to the oracle Bernstein width. 
\setcounter{equation}{12}
\begin{equation} \label{eq:13}
    \text{M}^{\text{PrPl-EB}}_t (m) := \prod_{i=1}^{t} \exp\left( \lambda_i (X_i - m) - v_i \psi_E(\lambda_i) \right),
\end{equation}
where 
$
    v_i := 4 (X_i - \hat{\mu}_{i-1})^2
$
and
\begin{equation} \label{eq:14}
    \psi_E(\lambda) := \frac{-\log(1 - \lambda) - \lambda}{4} \quad \text{for} \ \lambda \in [0, 1).
\end{equation}
\renewcommand{\thetheorem}{2} 
\begin{theorem}[Predictable plug-in empirical Bernstein CS (PrPl-EB)] \label{thm:PrPl-EB}
Suppose \( (X_t)_{t=1}^{\infty} \sim P \) for some \( P \in P^{\mu} \). For any \( (0,1) \)-valued predictable sequence \( (\lambda_t)_{t=1}^{\infty} \),  \\

$ C_t^{\text{PrPl-EB}} :=
   \left( \frac{\sum_{i=1}^{t} \lambda_i X_i}{\sum_{i=1}^{t} \lambda_i}
    \pm \frac{\log (2/\alpha) + \sum_{i=1}^{t} v_i \psi_E(\lambda_i)}{\sum_{i=1}^{t} \lambda_i} \right)$ \\
    
forms a \( (1 - \alpha) \)-CS for \( \mu \),  \\

as does its running intersection, \( \bigcap_{i \leq t} C_i^{\text{PrPl-EB}} \).\\

They recommend the predictable plug-in sequence \( (\lambda_t^{\text{PrPl-EB}})_{t=1}^{\infty} \) given by
\begin{equation} \label{eq:15}
    \lambda_t^{\text{PrPl-EB}} := 
    \sqrt{\frac{2 \log(2/\alpha)}{\hat{\sigma}^2_{t-1} t \log(1 + t)}} \
    \wedge \ c,  \quad
    \hat{\sigma}^2_t := \frac{\frac{1}{4} + \sum_{i=1}^{t} (X_i - \hat{\mu}_i)^2}{t+1},  
    \quad \hat{\mu}_t := \frac{\frac{1}{2} + \sum_{i=1}^{t} X_i}{t+1}.
\end{equation}
for some \( c \in [0,1) \) (a reasonable default is \( 1/2 \;\text{or}\; 3/4 \)).
\end{theorem}

\noindent \textbf{Remark 1} Theorem 2 yields computationally and statistically efficient empirical Bernstein-type confidence intervals (CIs) for a fixed sample size \( n \). Recalling \eqref{eq:15}, they recommend using \( \bigcap_{i \leq n} C_i^{\text{PrPl-EB}} \) along with the predictable sequence  
\begin{equation} \label{eq:16}
    \lambda_t^{\text{PrPl-EB}(n)} := 
    \sqrt{\frac{2 \log(2/\alpha)}{n \hat{\sigma}^2_{t-1}}} \wedge c.
\end{equation}
They call the resulting confidence interval the ‘predictable plug-in empirical Bernstein confidence interval’ or \textbf{[PrPl-EB-CI]} for short. \\

\noindent In i.i.d. settings, the width of [PrPl-EB-CI] scales with the true (unknown) standard deviation:  
\begin{equation} \label{eq:17}
    \sqrt{n} \left( \frac{{\log (2/\alpha) + \sum_{i=1}^{n} v_i \psi_E(\lambda_i)}}{\sum_{i=1}^{n} \lambda_i} \right)
    \xrightarrow{\text{a.s.}} 
    \sigma \sqrt{2 \log (2/\alpha)}.
\end{equation}

\noindent \eqref{eq:17} is the same asymptotic behavior that one would observe for CIs based on
Bernstein’s or Bennett’s inequalities. This is in contrast to the empirical Bernstein CIs of Maurer and
Pontil, whose limit would be \( \sigma \sqrt{2 \log (4/\alpha)}\).

\section{The Hedged Capital CS and CI}
\setcounter{equation}{23}
\begin{equation} \label{eq:24}
\begin{aligned}
    K^\pm_t (m) &:= \max \left\{ \theta K^+_t (m), (1 - \theta) K^-_t (m) \right\}, \\
    \text{where} \quad K^+_t (m) &:= \prod_{i=1}^{t} \left( 1 + \lambda^+_i (m) \cdot (X_i - m) \right), \\
    \text{and} \quad K^-_t (m) &:= \prod_{i=1}^{t} \left( 1 - \lambda^-_i (m) \cdot (X_i - m) \right).
\end{aligned}
\end{equation}
with \( \theta, m \in [0,1] \).

\begin{theorem}[Hedged capital CS (Hedged)]  
Suppose \( (X_t)_{t=1}^{\infty} \sim P \) for some \( P \in P^{\mu} \). Let \( (\tilde{\lambda}^+_t)_{t=1}^{\infty} \) and \( (\tilde{\lambda}^-_t)_{t=1}^{\infty} \) be real-valued predictable sequences not depending on \( m \), and for each \( t \geq 1 \) define  
\begin{equation}  
\lambda^+_t (m) := |\tilde{\lambda}^+_t| \wedge \frac{c}{m}, \quad  
\lambda^-_t (m) := |\tilde{\lambda}^-_t| \wedge \frac{c}{1 - m},  
\end{equation}  
for some \( c \in [0,1) \) (some reasonable defaults being \( c = 1/2 \) or \( 3/4 \)). Then\\

$ B^\pm_t := \{ m \in [0,1] : K^\pm_t (m) < 1/\alpha \} $ forms a \( (1 - \alpha) \)-CS for \( \mu \), \\

as does its running intersection \( \bigcap_{i \leq t} B^\pm_i \). Furthermore, \( B^\pm_t \) is an interval for each \( t \geq 1 \).  
\end{theorem}  
It is recommended to set \( \tilde{\lambda}^+_t = \tilde{\lambda}^-_t = \lambda^{\text{PrPl}\pm}_t \) as
\begin{equation} \label{eq:26}
   \lambda^{\text{PrPl}\pm}_t := \sqrt{\frac{2 \log (\frac{2}{\alpha})}{\hat{\sigma}^2_{t-1}t \log (t + 1)}}, \quad \hat{\sigma}^2_t := \frac{\frac{1}{4} + \sum_{i=1}^{t} (X_i - \hat{\mu}_i)^2}{t + 1}, \quad \hat{\mu}_t := \frac{\frac{1}{2}+ \sum_{i=1}^{t} X_i}{t + 1},
\end{equation}
for each \( t \geq 1 \), and truncation level \( c := \frac{1}{2} \, \text{or} \, \frac{3}{4}
 \). \\
 
\noindent \textbf{Remark 3} \quad Theorem 3 yields tight hedged CIs for a fixed sample size \( n \). Recalling (26), it is recommended to use \( \bigcap_{i \leq n} B^\pm_i \), and setting \( \tilde{\lambda}^+_t = \tilde{\lambda}^-_t = \tilde{\lambda}^\pm_t \) given by
\begin{equation} \label{eq:27}
    \tilde{\lambda}^\pm_t := \sqrt{\frac{2 \log( \frac{2}{\alpha})}{n \hat{\sigma}^2_{t-1}}}.
\end{equation}
The resulting CI is referred as the ‘hedged capital confidence interval’ or \textbf{[Hedged-CI]} for short. \\

\noindent Bets can be chosen such that hedged capital CSs and CIs converge at the optimal rates of \( O \left( \sqrt\frac{\log \log t}{t} \right) \) and \( O \left( \frac{1}{\sqrt{n}} \right) \) respectively. \\ 

\noindent The recommendations above don't satisfy the assumptions proving the convergence rate of the Hedged CIs. Yet, the above recommendations tighten the CIs (found through simulations), especially in low-variance, asymmetric settings.

\bibliographystyle{plain}
\bibliography{FJH25}
\end{document}
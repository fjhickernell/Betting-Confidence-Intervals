\documentclass{article}
\usepackage{amsmath,amsfonts,amssymb}
\usepackage[margin=1.3in]{geometry} % hack to have no pseudocode page break
\usepackage{url}
\title{Seeds for our simulations}
\date{March 2025}
\author{A.B.Owen}

\newcommand{\real}{\mathbb{R}}

\newcommand{\tab}{\hspace*{0.6cm}}
\newcommand{\qmc}{\mathrm{qmc}}
\newcommand{\mc}{\mathrm{mc}}
\begin{document}
\maketitle

Chapter 3.2 of \url{https://artowen.su.domains/mc/}
has about 4 pages on setting up random seeds and streams.
We don't need the full complexity.

It could work to just set the seed once at the beginning
for all computations.  It is important to do at least that
so that exactly the same computations can be obtained
later. That helps with intermittent bugs and reproducibility.
There might be an
advantage in making the seed depend on some of the
variables.  That can make some of the values more
reproducible if we change the set of dimensions or sample
sizes or the number of iterations.  It could also introduce
hard to notice bugs.

\medskip
The ridge functions are the most complicated
simulations. Here is pseudocode for the simulation that I hope is computationally
feasible for them.  I think that there is more than one reasonable way
to order the loops.  Here it is with just one seed for the whole simulation.
Below I discuss multiple seeding.
\medskip

\par\noindent
seed $\gets$ starting seed\\
dset $\gets$ set of dimensions $d$\\
gset $\gets$ set of ridge functions $g$\\
Nset $\gets$ set of budget sizes $N$\\
M $\gets$ number of repeated confidence interval constructions\\
$D\gets$ max(dset)\\
For $N\in\mathrm{N set}$\\
\tab For $n\in\mathrm{n set(N)}$\\
\tab\tab $R=N/n$\\
\tab\tab Make QMC point set $X^\mathrm{QMC} \gets \qmc(M,R,n,D)\in\real^{M \times R \times  n\times D}$\\
\tab\tab Make MC point set $X^\mathrm{IID} \gets \mc(M,R,n,D)\in\real^{M \times R \times n\times D}$\\
\tab\tab For $d\in\mathrm{dset}$\\
\tab\tab\tab For $m=1,\dots,M$\\
\tab\tab\tab\tab For $r=1,\dots,R$\\
\tab\tab\tab\tab\tab $X(\qmc)\gets X^\mathrm{QMC}_{m,r,0:n,0:d}$ \\
\tab\tab\tab\tab\tab $X(\mc)\gets X^\mathrm{IID}_{m,r,0:n,0:d}$\\
\tab\tab\tab\tab\tab For $g\in\mathrm{gset}$\\
\tab\tab\tab\tab\tab\tab $Y(\qmc,N,m,n,r,d,g)\gets$ average $g$ on $X(\qmc)$\\
\tab\tab\tab\tab\tab\tab $Y(\mc,N,m,n,r,d,g)\gets$ average $g$ on $X(\mc)$\\
\tab\tab\tab Wbet$(\qmc,N,m,n)\gets$ betting qmc width\\
\tab\tab\tab Wbet$(\mc,N,m,n)\gets$ betting mc width\\
\tab\tab\tab WeB$(\qmc,N,m,n)\gets$ empirical Bernstein qmc width\\
\tab\tab\tab WeB$(\mc,N,m,n)\gets$ empirical Bernstein mc width\\
Write all the widths to a file

\bigskip

In principle we could reset the seed
for every $N,m,n,r$ for the QMC points
and also for the MC points.  Then there
would be a seed setting call just before each place
where point sets $X$ are made.
That raises the issue of coming up with all
the seeds you need.  You can start with an
array whose dimensions correspond to all of the
seed combinations that you will need.  Then
write seeds into that array.  It could work to
just write seeds 1 through K to that array
where K is the number of entries.  Or you could
make the seed be some function of the arguments
to that array like $\log_2(N)+100*m+10000*n+100000*r$.
That gets tricky because you might overflow one
of the counts and find the same seed used twice.
Another approach is to set up the array and
then fill it with integers $1$ to $K$ in random
order.  Or make a huge array of seeds and take
a random sample of $K$ distinct ones and put 
those into your array.

In a first run, it would be smart to start all the
loops at some small number of values (say 2 levels each)
and to keep appending the widths to a file as soon 
as they are computed. That way
you can see what is going on. You can also time
the runs and see how the run time grows, watching
for any places where it is nonlinear. Also after ramping
up you might later have to kill the simulation and
it would be useful to have the intermediate results
to study then.

Regarding seeds, it would be nice to have a separate
seeds for each N, m, n and for qmc vs mc.  I would
rate that as a nice to have, not an essential.  It is
not necessary to have separate seeds for each $r$, 
though I don't see how it would hurt.  Trying to make
the seed depend only on N, m, n, qmc/mc but not on r
would involve some tricky seed saving and resetting
and I think that would be too fussy and error prone.

One advantage of complex seeding is that when you
ramp up the computations from small ones to large
ones some or even all of the computed widths
will stay the same.  It is reassuring to see the
same numbers come up. 

I would start with just one seed being set while
looking at ways to time the tiny versions of the
simulation.  You might learn something about how
to speed up the simulations that would require
changing how you do the seeds.

The most expensive runs will be for the largest
values of $N$.  It might become necessary to have
$M$ become $M(N)$ with fewer iterations for the
very largest value of $N$. E.g, 20 for the biggest
$N$ and 100 for all the others.  Because the intersesting
values of $N$ are powers of 2 it is common for the
biggest $N$ to be about half of the total computation.

\end{document}